\chapter{Obiectivele proiectului}\label{ch:obiective}
\pagestyle{fancy}
Obiectivul acestui proiect este de a oferi utilizatorului un sistem precis si scalabil pentru monitorizarea in timp real a parametrilor de temperatura si 
umiditate dintr-un mediu. Sistemul integreaza achizitia de date precise de la senzori cu utilizarea unor protocoale de incredere pentru comunicarea si 
stocarea datelor care, in final, sunt puse la dispozitia utilizatorului intr-o platforma de vizualizare si control al datelor usor accesibila. Platforma este 
bazata pe sistemul de operare Android si poate fi accesata de la distanta din orice locatie in care internetul este accesibil. De asemenea, platforma ofera 
o reprezentare grafica a datelor din trecut pentru a facilita analiza acestora.

\section{Cerinte functionale}\label{sec:cerinte_functionale}
\subsection{Interfata cu utilizatorul}\label{subsec:cf_interfata}
\begin{itemize}
    \item utilizatorul trebuie sa poata instala senzori noi
    \item utilizatorul trebuie sa poata sterge un senzor din lista 
    \item utilizatorul trebuie sa fie ghidat in timpul instalarii unui nou senzor cu sfaturi si informatii utile
    \item utilizatorul trebuie sa poata vizualiza toti senzorii instalati sub forma de lista
    \item utilizatorul trebuie sa poata accesa la alegere un senzor din lista si sa obtina informatii despre acesta
    \item fiecare senzor trebuie sa afiseze statusul conectivitatii in dreptul numelui direct in lista
    \item fiecare senzor trebuie sa aiba pagina lui proprie in care sunt afisate detaliile acestuia si datele achizitionate 
    \item datele de temperatura si umiditate trebuie actualizate in timp real si trebuie sa afiseze unitatea de masura specifica fiecarei valori
    \item datele istorice trebuie afisate sub forma grafica
    \item utilizatorul trebuie sa poata vizualiza marcajul temporal si valorile exacte ale metricilor prin selectarea unui punct din grafic.
    \item utilizatorul trebuie sa poata alege perioada pe care vizualiza datele in grafic
    \item utilizatorul trebuie sa poata alege intervalul de achizitionare al datelor
    \item utilizatorul trebuie sa poata seta parametrii de receptie ai alarmelor
    \item utilizatorul trebuie sa poata reseta senzorul la setarile din fabrica prin mentinerea apasata a unui buton timp de 10 secunde 
    \item senzorul trebuie sa ofere o lumina led care sa indice statusul lui pe parcursul instalarii.
\end{itemize}

\subsection{Transmiterea datelor}\label{subsec:cf_transmitere}
Sistemul trebuie sa transmita datele catre utilizator si unitatea centrala utilizand un protocol fara fir, de exemplu Wi-Fi, si sa asigure integritatea 
datelor prin utilizarea unui protocol de comunicatie bazat pe cerere si raspuns.
 
\subsection{Colectarea datelor}\label{subsec:cnf_colectare}
Sistemul trebuie sa colecteze date de la toti senzorii si sa asigure respectarea intervalelor specificate de catre fiecare utilizator.

\subsection{Salvarea datelor}\label{subsec:cnf_salvare}
Sistemul trebuie sa adauge o unitate de timp la datele achizitionate de la senzori si sa le salveze intr-o baza de date adecvata pentru tipul si 
structura acestora si sa asigure interogarea eficienta.

\section{Cerinte non-functionale}\label{sec:cerinte_nonfunctionale}
\subsection{Performanta}\label{subsec:cnf_performanta}
Timpul de raspuns al sistemului este esential pentru monitorizarea in timp real al datelor si pentru receptionarea de alerte. Durata acestuia trebuie 
sa fie de sub o secunda si este compusa din durata de achizitie, procesare si transmitere a datelor catre unitatea centrala, durata de procesare si 
redirectionare catre aplicatia Android si durata de procesare si afisare a datelor de catre aplicatie. Aceasta metrica este in stransa relatie cu 
utilizarea protocolului de comunicatie MQTT specializat pe timpul de raspuns scurt. 

In cazul afisarii datelor din trecut, timpul de raspuns al sistemului este compus din durata de transmisie a unei cereri, procesarea acesteia si returnarea
datelor cerute. Sistemul asigura o durata scurta de raspuns prin controlul debitului datelor. Acest control este realizat prin afisarea grafica a metricilor
de temperatura si umiditate care presupune afisarea unei serii restranse de date, de exemplu, datele care au fost acumulate in 12 ore.

Latenta sistemului este definita de perioada dintre inceputul achizitiei temperaturii si a umiditatii si afisarea datelor in aplicatia Android. Pentru 
monitorizarea datelor atat in timp real cat si istorice aceasta metrica este foarte importanta, deoarece o schimbare radicala in valorile extrase din mediu 
nu mai este relevanta daca de la momentul in care a fost extrasa si pana la momentul in care utilizatorul o observa a trecut mult timp.

\subsection{Scalabilitate}\label{subsec:cnf_scalabilitate}
Scalabilitatea sistemului consta in numarul de senzori care pot fi conectati la un utilizator si in numarul de utilizatori care pot fi administrati 
in acelasi timp fara a afecta performanta sistemului. Acesta trebuie sa utilizeze un protocol de comunicatie care necesita un trafic de date mic pentru 
mentinerea conexiunii si care are un volum mic de date utile pentru a asigura un nivel de scalabilitate crescut.

\subsection{Utilizabilitate}\label{subsec:cnf_utilizabilitate}
Aplicatia trebuie sa ofere utilizatorului o experienta placuta si facila. Utilizatorul trebuie sa fie ghidat si sa aiba la dispozitie informatiile necesare
instalarii unui senzor nou fara sa fie necesare cunostinte tehnice. Vizualizarea statusului conexiunii si a datelor trebuie sa fie intuitiva.

\subsection{Administrarea datelor}\label{subsec:cnf_administrare}
Sistemul trebuie sa ofere o structura a bazei de date care sa asigure pastrarea datelor pe durate de timp foarte mari, de ordinul anilor, si care sa 
raspunda rapid si eficient interogarilor de date de la utilizatori. Utilizarea unei baze de date de tip serie in timp este potrivita in acest context, 
deoarece utilizeaza tehnici de reordonare si restructurare bazate pe cel mai utilizat tip de interogare.

\subsection{Mentenabilitate}\label{subsec:cnf_mentenabilitate}
Scalabilitatea verticala este reprezentata de capacitatea de dezvoltare a sistemului prin inbunatatirea software-ului sau prin adaugarea de resurse 
suplimentare. Acest tip de scalabilitate este obtinut prin utilizarea unui sistem distribuit in care fiecare componenta are o sarcina bine definita 
si un volum scazut ceea ce face ca timpul necesar operatiunilor de modificare sau imbunatatire sa fie scurt.

\section{Cazuri de utilizare}\label{sec:cazuri_de_utilizare}
\subsection{Instalare senzor}\label{subsec:cdu_instalare}
Acest caz de utilizare are loc la achizitia unui nou senzor sau la resetarea unui senzor actual:
\begin{enumerate}
    \item Deschiderea aplicatiei
    \item Accesarea butonului de instalare senzor
    \item Parcurgerea pasilor de instalare descrisi direct in aplicatie
    \item Vizualizarea senzorului in lista
\end{enumerate}
\subsection{Monitorizare date}\label{subsec:cdu_monitorizare}
\begin{enumerate}
    \item Deschiderea aplicatiei
    \item Cautarea in lista a senzorului pentru care se doreste Vizualizarea
    \item Accesarea senzorului dorit
    \item Vizualizare date curente in timp real
    \item Vizualizare date istorice sub forma de grafic
\end{enumerate}