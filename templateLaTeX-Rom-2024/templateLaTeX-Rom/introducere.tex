\chapter{Introducere}\label{ch:intro}
\pagestyle{fancy}
Dezvoltarea rapida a tehnologiei a dus la necesitatea automatizarii sistemelor de mentenanta si monitorizare a echipamentelor dintr-o gama larga de 
domenii, de la procese de fabricatie si productie industriale pana la un stil de viata mai confortabil si protejat. Mentinerea unui mediu optim pentru
echipamente de inalta precizie aduce un avantaj economic semnificativ prin reducerea frecventei si a costului mentenantei periodice. Sistemul in timp
real permite detectarea unei anomalii intr-un timp foarte scurt, ceea ce permite interventia rapida si evitarea unui eveniment nedorit.

Spectrul domeniilor de aplicabilitate ale unui sistem de tipul internetul lucrurilor pentru monitorizarea temperaturii si a umiditatii este foarte 
larg. Sistemele si componentele care ne fac viata mai usoara, cum ar fi sistemele hidraulice utilizate in productie sau serverele care ne mentin 
interconectati, necesita valori ale temperaturii si ale umiditatii intre anumite limite pentru a asigura functionarea pe o durata cat mai lunga.

Internetul lucrurilor reprezinta o retea vasta alcatuita din dispozitive fizice inteligente. Un dispozitiv face parte din aceasta retea daca contine
echipamentul hardware si tehnologia necesara pentru: a se conecta la internet printr-un protocol wireless, a observa sau interactiona cu mediul si
a interschimba date cu o unitate centrala. Aceste sisteme se regasesc intr-un numar tot mai mare de obiecte care ne inconjoara, de exemplu,
electrocasnicele care iti permit controlul de pe telefon si care te anunta ca ciclul de functionare s-a incheiat.

Sistemele traditionale de monitorizare a temperaturii si umiditatii care furnizau datele utilizatorului prin cablu de retea sau printr-un afisaj
electronic au fost inlocuite cu dispozitive care fac parte din internetul lucrurilor, deoarece acestea din urma prezinta cateva avantaje importante.
Costul de instalare al senzorilor pentru monitorizarea facilitatilor industriale mari este redus, deoarece cantitatea de cablu care trebuie
rutat prin intreaga facilitate este redusa semnificativ sau complet, in cazul dispozitivelor alimentate pe baterie. De asemenea, rutarea cablurilor 
printr-o astfel de facilitate nu este posibila sau este foarte dificila din cauza elementelor in miscare sau a acesului fizic restrictionat, de exemplu,
nacela unei turbine eoliene. Dispozitivele wireless ofera acces foarte usor la datele senzorilor si prezinta un nivel inalt de scalabilitate si 
flexibilitate atat in ceea ce priveste intalarea cat si in accesul datelor.

Monitorizarea in timp real permite adaptarea imediata a parametrilor sistemului la schimbarile mediului optimizand functionarea acestuia, de exemplu,
reducerea consumului de resurse al sistemelor de incalzire si racire dintr-o cladire de birouri. De asemenea, monitorizarea continua permite notificarea 
utilizatorilor asupra schimbarilor care se produc in mediu, aceste notificari pot fi simple instiintari sau alarme. In cazul celor din urma, interventia 
cat mai rapida a echipei de mentenanta poate preveni sau reduce daunele cauzate de un eveniment nedorit.

Accesul facil la date istorice ofera posibilitatea de a efectua cercetari detaliate pentru detectarea de trenduri si perioade repetitive, pe baza
carora se pot prezice si imbunatati perioadele care urmeaza. De asemenea, deschide oportunitatea catre utilizarea inteligentei artificiale pentru 
a invata pe baza parametrilor trecuti si a se adapta la viitor in mod automat.

Imbunatatirea stilului de viata este una dintre principalele tinte ale fiecarei persoane. Monitorizarea continua a temperaturii si a umiditatii 
dintr-o locuinta permite mentinerea acestor parametrii la valori optime pentru confortul si sanatatea fiecarei persoane. Valoarea optima a umiditatii 
%%(TODO: https://med.ro/pneumo/umiditate-optima/)%% 
intr-o incapere este cuprinsa intre 30\% si 50\%, limita maxima fiind de 60\%, iar a temperaturii in timpul 
somnului este de 20.
%%(TODO degree) (TODO: https://www.ncbi.nlm.nih.gov/pmc/articles/PMC3427038/)%%. 
Mentinerea parametrilor intre aceste valori reduce factorii de risc in aparitia a numeroase boli, in special cele respiratorii.

