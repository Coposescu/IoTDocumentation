\chapter{Introducere}\label{ch:intro}
\pagestyle{fancy}
Dezvoltarea rapida a industriei si aglomerarea mediilor urbane a dus la un nivel ridicat de poluare a aerului care este in continua crestere. Poluarea aerului 
a devenit una dintre cele mai mari provocari in ceea ce priveste mediul inconjurator si starea de sanatate a populatiei. Observarea consecintelor cauzate de 
o calitate slaba a aerului din mediul inconjurator a facut ca atentia populatiei si a entitatilor guvernamentale sa se indrepte catre sisteme de monitorizare 
a calaitatii aerului. Lucrarea de fata aduce o contributie spre solutionarea problemei poluarii aerului.

Cercetarea stiintifica in domeniul calitatii aerului mediului inconjurator este in continua crestere, oferind senzori din ce in ce mai performanti pentru 
detectarea si masurarea poluantilor aerului. Aceasta stiinta combinata cu tehnologia Internetul Lucrurilor (IoT) creeaza o solutie inovativa pentru 
monitorizarea calitatii aerului. Aceasta solutie este compusa dintr-o retea de senzori interconectati care masoara gradul de prezenta a poluantilor din aer si 
care transmit aceste date la un punct central aflat in cloud utilizand tehnologia wireless. Astfel, prin aplicatii mobile sau web care ofera diferite moduri de 
vizualizare a datelor, este posibila observarea si analiza datelor istorice sau in timp real de la un numar foarte mare de senzori amplasati oriunde in lume. 
Indiferent de amplasarea fizica a senzorilor wireless, in medii urbane, in medii industriale sau in spatii inchise, acestia ofera un avantaj semnificativ in 
fata metodelor traditionale de monitorizare a calitatii aerului oferind o acoperire spatiala mult mai mare si observare in timp real a datelor.

Sistemele traditionale de monitorizare a calitatii aerului prezinta o plaja mai larga a poluantilor prezenti in aer si ofera capacitatea de identificare a acestora 
punctual prin utilizarea a diferite metode chimice. Acest tip de monitorizare a calitatii aerului este limitat de costuri mari si locatii fixe. In mod uzual, 
datele colectate de asfel de sisteme sunt prezentate ca o medie pe o perioada indelungata ceea ce duce la incapacitatea de observare a unor evenimente care cauzeaza 
scaderea calitatii aerului pentru o perioada scurta de timp si care pot necesita o interventie. In timp ce sistemele tranditionale in continuare reprezinta solutia potrivita 
pentru medii limitate din punct de vedere spatial si care necesita detectarea punctuala a unui numar mare de poluanti, sistemele de monitorizare a calitatii aerului 
care ofera capacitatea de observare a datelor in timp real, de oriunde in lume si pe o arie de acoperire foarte mare reprezinta solutia pentru factorii principali 
de poluare care afecteaza intreaga populatie precum, emisiile rezultate in urma arderii carburantilor, activitatile facilitatior industriale, incendii etc.

Un sistem de monitorizare a calitatii aerului bazat pe tehnologia IoT aduce multe beneficii aplicabile in diferite domenii. Amplasarea unui astfel de sistem intr-o 
locuinta contribuie la cresterea calitatii vietii persoanelor care traiesc in acea locatie. Activitatile zilnice precum gatitul sau periodice precum renovarea 
determina o crestere a poluantilor din aer precum compusii organici volatili sau particulele in suspensie. Aceasta crestere poate depasi limita de siguranta, iar 
expunerea indelungata poate duce la aparitia unor probleme de sanatate. Solutia pentru aceste probleme este aerisirea corecta a spatiului. De asemenea, o crestere 
constata a poluantilor din aer pe o perioada mai indelungata in care nu s-a actionat asupra mediului descopera existenta unei surse poluante in locuinta, cum ar fi 
mucegaiul. Datele in timp real si datele istorice oferite de un astfel de sistem ajuta la detectarea unor astfel de probleme.

Monitorizarea calitatii aerului in spatiile inchise mari precum cladirile de birouri, scolile sau spitalele, ajuta la mentinerea unui mediu sanatos si 
productiv pentru angajatii si personalul acestora. Datele in timp real permit observarea si interventia imediata in cazul unor evenimente nedorite care 
pot fi costisitoare. O analiza a datelor istorice poate ajuta la detectarea functionarii incorecte a sistemului de incalzire, racire si ventilare a acestor 
cladiri care poate avea un impact economic semnificativ. De exemplu, functionarea sistemului in perioadele in care cladirile sunt goale.

Nivelul ridicat de scalabilitate al unui astfel de sistem permite acoperirea unei zone foarte mari la un cost mult sub cel al unui sistem traditional. Impanzirea 
unei facilitati industriale cu astfel de senzori ajuta la mentinerea calitatii aerului in intervalele impuse de reglementarile de mediu prin continua monitorizare 
a emisiilor si alertarea personalului in cazul incalcarii reglementarilor. Acest lucru permite interventia imediata si evitarea unui posibil eveniment nedorit.

La baza unui sistem de monitorizare a calitatii aerului se afla reteaua de senzori care colecteaza date in mod continuu despre indicii cheie de poluare a mediului 
cum ar fi particulele in suspensie (PM) si compusii organici volatili (VOC). Pe langa indicii de poluare a mediului senzorii colecteaza si date de temperatura si umiditate 
care sunt necesare pentru determinarea modului in care substantele poluante sunt dispersate in aer. Combinatia dintre aceasta retea de senzori si tehnologia IoT 
ofera access la o baza de date foarte mare care, analizata corespunzator, ofera informatii care nu erau acesibile pana acum. Prin aplicarea de algoritmi de filtrare 
si predictibilitate pe aceste date se pot observa modele repetitive in timp care ajuta la intelegerea mai profunda a factorilor care polueaza mediul.

Integrarea sistemelor de monitorizare a aerului din intreaga lume intr-o patforma publica ajuta la constientizarea populatiei in legatura cu importanta calitatii 
aerului din mediul inconjurator si automat la imbunatatirea acestuia pe termen lung. De asemenea, o astfel de platforma ajuta populatia in luarea de decizii. 
De exemplu, o vacanta intr-o zona cu un indice scazut al calitatii aerului poate determina individul sa ajusteze planul de vacanta prin schimbarea zonei sau prin 
reducerea perioadei.

Provocarile sau punctele principale de dezvoltarea ale unui sistem IoT de monitorizare a calitatii aerului sunt acuratetea si increderea in datele obtinute 
de la senzori. Exista un compromis intre costul senzorului si capabilitatile acestuia, in special in cazul instalarilor de retele mari de senzori. Necesitatea in 
crestere de astfel de senzori duce la cresterea cercetarilor efectuate in acest domeniu, iar acest lucru va aduce cu sine imbunatatirea si scaderea costului acestora. 
De asemenea, calibrarea adecvata a senzorilor si respectarea conditiilor de instalare a acestora este foarte importanta pentru increderea in datele obtinute.

In capitolele urmatoare se vor prezenta obiectivele pe care acest proiect isi propune sa le atinga, studiul bibliografic al domeniului, descrierea teoretica 
a tehnologiilor utilizate in realizarea proiectului, descrierea detaliata a implementarii pe module, testarea si validarea sistemului si concluziile la care s-a 
ajuns in urma dezvoltarii acestui sistem.

