\chapter{Proiectare de detaliu și implementare}
\pagestyle{fancy}

Acest capitol prezinta in detaliu solutia propusa pentru un sistem de monitorizare a calitatii aerului. Aceasta include protocoale de 
comunicatie, limbaje de programare, arhitecturi de abstractizare si senzorii utilizati pentru masurarea indicilor de calitate a aerului.

\section{Arhitectura sistemului}\label{sec:pi_general_architecture}
Arhitectura generala a sistemului este compusa din 5 componente pricipale interconectate pentru a crea o retea bine definita si cu o flexibilitate
ridicata:
\begin{enumerate}
	\item Telefon Mobil
	\item Broker MQTT
	\item Server RESTful
	\item Baza de date MongoDB
	\item Senzor
\end{enumerate}

\

Figura \ref{fig:ArhitecturaGenerala} prezinta arhitectura generala a sistemului continand modulele principale ale acestuia si protocoalele de comunicatie 
dintre acestea.
\begin{figure}[H]
    \centering
    \includegraphics[scale=0.68]{figs/ArhitecturaGenerala.png}
    \caption{Arhitectura generala a proiectului}
    \label{fig:ArhitecturaGenerala}
\end{figure}

In intreaga retea este utilizat fusul orar UTC (Timpul Universal Coordonat) pentru a evita diferentele de fus orar local.

In subcapitolele urmatoare va fi descris in mod detaliat fiecare modul al sistemului si protocoalele de comunicatie utilizate pentru comunicarea dintre 
acestea.

\section{Modulul Telefon Mobil}\label{sec:pi_appandroid}
Acest modul este un telefon mobil cu sistem de operare Android pe care a fost dezvoltata o aplicatie. Aplicatia Android reprezinta interfata cu utilizatorul oferindui 
acestuia un mod usor de a gestiona mai multi senzori si de a monitoriza datele venite de la acestia. Este scrisa in limbajul de programare Java si in mediul de dezvoltare 
integrat Android Studio IDE.

\subsection{Diagrama de clase}\label{subsec:pi_appandroid_digrama_clase}
In sistemul de operare Android clasele care definesc o interfata grafica sunt denumite activitati. La deschiderea aplicatiei este deschis firul de executie principal 
care are sarcina de a afisa interfetele grafice si de a gestiona actiunile utilizatorului, de exemplu, apasare de buton. Fiecare clasa de tip activitate extinde 
clasa AppCompatActivity care ofera componente pentru afisarea grafica si metode callback pentru gestionarea navigarii intre mai multe activitati. 

Aplicatia este compusa din 3 activitati principale si o serie de activitati care au scopul de ghidare a utilizatorului prin procesul de instalare:
\begin{itemize}
	\item ActivityWelcome - reprezinta activitatea care este afisata la deschiderea aplicatiei.
	\item ActivitySensor - reprezinta activitatea care este afisata atunci cand utilizatorul selecteaza un senzor din lista.
	\item ActivityChart - reprezinta activitatea in care sunt afisate grafic datele de la senzor.
	\item ActivityInstall - reprezinta un grup de activitati secundare afisate atunci cand utilizatorul instaleaza un senzor nou. Fiecare dintre aceste 
	activitati reprezinta un pas din procesul de instalare.
\end{itemize}

\

Figura \ref{fig:AndroidClassDiagram} prezinta diagrama de clase a aplicatiei Android.
\begin{figure}[H]
    \centering
    \includegraphics[scale=0.68]{figs/AndroidClassDiagram.png}
    \caption{Arhitectura generala a proiectului}
    \label{fig:AndroidClassDiagram}
\end{figure}

\subsection{Procesul de initializare a aplicatiei}\label{subsec:pi_appandroid_initializare}
La incarcarea primei activitati care este afisata pe ecranul dispozitivului mobil, numita ActivityWelcome, se extrage din memorie o lista de senzori care au fost 
instalati in prealabil. Daca nu exista senzori instalati, aceasta lista nu va contine elemente. Lista este parcursa si pentru fiecare senzor este pornit un nou 
fir de executie periodic. Acest fir de executie are rolul de a gestiona conexiunea cu modulul MQTT Broker pentru senzorul respectiv, iar periodicitatea 
acestuia perimite verificarea si reincercarea conexiunii la un interval fix de timp. Apoi aceasta lista este afisata in interfata grafica. Fiecare element din lista 
contine: denumirea senzorului, tipul de senzor, statusul conexiunii si o imagine reprezentativa a senzorului.

\subsection{Memorarea senzorilor instalati}\label{subsec:pi_appandroid_memorare_senzori}
Pentru memorarea senzorilor installati este utilizata biblioteca SharedPreferences. Aceasta contine rutine pentru salvarea datelor intr-un fisier din memoria 
dispozitivului mobil intr-un format de forma (cheie, valoare). Primul camp din acest fisier reprezinta numarul de senzori salvati, iar campurile ce urmeaza 
reprezinta senzorii instalati. Aceasta biblioteca este potrivita pentru memorarea datelor sub forma (cheie, valoare) si pentru compatibilitatea cu versiuni mai 
vechi de Android, spre deosebire de alte biblioteci precum DataStore care sunt potrivite pe seturi de date complexe si functioneaza doar in versiunile mai noi de 
Android.

\subsection{Fire de executie periodice}\label{subsec:pi_appandroid_fire_executie}
Pentru firele de executie periodice este utilizata biblioteca ScheduledThreadPoolExecutor. Acesasta permite crearea unui grup de fire de executie care se executa
in paralel, spre deosebire de biblioteca Timer care are un singur fir de executie, iar o sarcina de durata mai lunga poate intarzia alte sarcini care asteapta 
executia. De asemenea, in cazul unei erori, doar sarcina in care a aparut eroarea va fi oprita, celelalte sarcini fiind executate in mod obisnuit. Aceasta 
biblioteca ofera siguranta continuarii executiei aplicatiei in cazul unei erori izolate.

\subsection{Transferul intre activitati}\label{subsec:pi_appandroid_transfer_activitati}
La selectarea de catre utilizator a unui senzor din lista afisata este deschisa o noua activitate, numita ActivitySensor, iar la atingerea butonului de adaugare a unui 
nou senzor este deschisa prima activitate din grupul de activitati specifice instalarii. Pentru navigarea la o noua activitate si pentru transferarea de informatii intre 
activitati este utilizata clasa Intent. Aceasta clasa reprezinta o descriere abstracta a unei operatii. Cea mai semnificativa utilizare a acesteia este pentru operatia de 
deschidere a unei noi activitati. Pentru aceasta operatie se apeleaza functia startActivity() care primeste ca parametru o instanta a clasei Intent care contine informatiile 
necesare pentru ca operatia sa fie executata. Informatii precum activitatea parinte si activitatea care urmeaza a fi executata sunt necesare, iar in plus pot fi adaugate 
informatii care sa fie transferate catre activitatea copil utilizant functia putExtra() care primeste ca parametru o structura de tipul (cheie, valoare).

\subsection{Conexiunea cu modulul MQTT Broker}\label{subsec:pi_appandroid_conectare_mqtt}
Pentru fiecare fir de executie specific unui senzor se creaza o instanta a clasei ScheduledDataAcquisition care implementeaza interfata runnable si executa periodic o rutina 
in care se verifica daca senzorul este conectat la modulul MQTT Broker. La prima executie senzorul este considerat deconectat si se executa rutina de conectare. 
Aceasta rutina utilizeaza clasa MQTTClient care ofera metodele necesare realizarii si gestionarii conexiunii cu modulul MQTT Broker. Pentru realizarea conexiunii se 
executa metoda connect a clasei MQTTClient care primeste ca paramterii 2 rutine callback. Aceste rutine definesc actiuni ce sunt executate atunci cand au loc 
diferite evenimente in tranzactionarea cu modulul MQTT Broker. Mai jos sunt enumerate cele mai importante astfel de evenimente:
\begin{itemize}
	\item Connectarea cu success la modulul MQTT Broker - cand are loc acest eveniment se executa rutina de subscriere pentru receptionarea in timp real a datelor de la 
	senzor.
	\item Pierderea conexiunii - acest eveniment va modifica statusul conexiunii din lista de senzori a-i activitatii ActivityWelcome si din ActivitySensor.
	\item Receptia unui mesaj - acest eveniment va apela metoda notifyListeners() a obiectului UIThreadHandler care va actualiza ultimele valori ale indicilor de 
    calitate a aerului afisate in activitati. De asemenea, acest eveniment va verifica si statusul conexiunii senzorului si il va schimba 
	daca este cazul.
\end{itemize}

\

Clasa MQTTClient utilizeaza o instanta a clasei MqttAndroidClient oferita de biblioteca eclipse.paho.mqttv3 care reprezinta o implementare asincrona a 
al unui client al protocolului MQTT si are rolul de a gestiona impachetarea messajelor in formatul protocolului MQTT si tranzactionarea acestora cu un server MQTT. De asemenea, 
clasa MQTTClient defineste rutine de gestionare a exceptilor ridicate de obiectul MqttAndroidClient in cazul unei erori. 

\subsection{Transferul de informatii intre firele de executie}\label{subsec:pi_appandroid_transfer_info_threads}
Clasa UIThreadHandler are rolul de a efectua modificari in interfetele grafice de pe un fir de executie extern. Clasele de tip activitate sunt executate pe un fir de executie 
care are rolul strict de a raspunde la actiunile utilizatorului si doar acest fir de executie poate face modificari in interfata grafica, iar interogarea bazei de date sau 
receptionarea de date de la modulul MQTT sunt executate pe fire de executie diferite. Aceasta clasa realizeaza transferul de date sau evenimente care necesita modificarea 
interfetei grafice si care au fost primite pe un fir de executie extern catre firul de executie al interfetei grafice. Pentru a realiza acest transfer, instanta clasei  
UIThreadHandler mentine o lista de clase de tip Listener. Activitatile AvtivityWelcome si ActivitySensor sunt clase de tip Listener, deoarece implementeaza interfata 
Listener si metoda updateMetrics() a acesteia, iar la creare se inregistreaza in lista de obiecte Listener mentinuta de instanta UIThreadHandler. Fiecare activitate 
defineste in metoda updateMetrics() ce anume va fi modificat in interfata grafica. Atunci cand sunt receptionate date de la modulul MQTT Broker, se executa metoda 
notifyListeners() a obiectului UIThreadHander. Aceasta metoda parcurge lista de clase de tip Listener si pentru fiecare apeleaza metoda updateMetrics(). Aceasta metoda nu este 
apelata direct, ci prin obiectul Handler care primeste ca parametru o rutina de tip Runnable si care este pus intr-o coada de executie a interfetei grafice.

\subsection{Modul de afisare al parametrilor de calitate a aerului}\label{subsec:pi_appandroid_afisare_parametrii_aqi}
La selectarea unui element din lista de senzori afisata in activitatea ActivityWelcome este creata activitatea ActivitySensor. Rolul acesteia este de a prezenta informatiile 
senzorului selectat si ultimele valori ale indicilor de calitate a aerului primite de la acesta sub forma unei grile care are 2 coloane si mai multe randuri. La creare 
se citesc din obiectul Intent primit de la activitatea ActivityWelcome informatiile senzorului intr-un obiect SensorListItem si se afiseaza informatiile senzorului in 
interfata grafica. 

La apasarea oricarui element din grila in care sunt prezentate ultimele valori citite de la sonzor se va deschide activitatea ActivityChart. Aceasta activitate are rolul 
de a afisa indicii de calitate a aerului sub forma grafica. La crearea activitatii se initializeaza cate un grafic pentru fiecare metrica si se interogheaza baza de date 
pentru valorile din ultimele 10 minute. La primirea valorilor citite din baza de date, acestea sunt procesate si afisate in graficul corespunzator. Daca nu exista date 
in ultimele 10 minute inseamna ca senzorul nu este conectat, iar statusul acestuia este modificat corespunzator.

\subsection{Conexiunea cu baza de date}\label{subsec:pi_appandroid_conexiunea_cu_db}
Pentru interogarea bazei de date este utilizata biblioteca Retrofit a carei functionalitate teoretica este descrisa in sectiunea \ref{sec:retrofit}. Pentru implementarea 
rutinelor de interogare a bazei de date utilizand aceasta biblioteca sunt utilizate urmatoarele clase din figura \ref{fig:AndroidClassDiagram}:
\begin{itemize}
	\item Clasa DBAPIClient - are rolul de a crea un obiect de tip Retrofit. Pentru crearea acestuia sunt necesare: adresa URL a server-ului, un obiect GsonConverterFactory 
	pentru convertirea automata a datelor si o instanta a clasei OkHttpClient.
	\item Interfata DBAPIInterface - declara metodele pentru interogarea bazei de date utilizand adnotari.
	\item Clasa SensorReading - este o clasa model care contine campurile receptionate in raspunsul metodei doGetSensorReading().
	\item Clasa SensorReadingsList - este o clasa model care contine o lista de obiecte de tip SensorReading. Aceasta lista reprezinta raspunsul metodei doGetSensorReadingsList().
\end{itemize}

La initierea unei interogari a bazei de date se obtine obiectul Retrofit utilizand metoda getClient() a clasei DBAPIClient. Se apeleaza metoda create() a obiectului Retrofit care 
primeste ca parametru interfata DBAPIInterface, iar pe baza acestei interfete, Retrofit va genera automat o clasa care contine implementarea metodelor declarate in aceasta. 
Utilizand instanta clasei creata automat se acceseaza una din metodele acesteia si se pune intr-o coada de transmisie impreuna cu o metoda callback care va fi executata cand 
este receptionat raspunsul de la server sau cand expira timpul de asteptare. La receptionarea raspunsului, biblioteca Retrofit va interpreta datele receptionate in format GSON 
bazat pe clasa model specifica metodei care s-a executat si va crea un obiect de acest tip. In metoda callback se citeste obiectul si se adauga valorile de temperatura si 
umiditate in graficul respectiv fiecareia. Pentru interpretare, a fost utilizat obiectul GsonConverterFactory care a fost instantiat la crearea obiectului Retrofit.

\subsection{Instalarea unui nou senzor}\label{subsec:pi_appandroid_instalare_senzor}
Pentru instalarea unui nou senzor, la apasarea butonului de adaugare din activitatea ActivityWelcome se deschide prima activitate din setul de activitati pentru instalare. 
Aceasta activitate cere utilizatorului inserarea datelor de conectare la senzor oferite in manualul de instalare. La apasarea butonului Next se deschide a activitatea de 
instalare specifica pasului doi in care se cere introducerea informatiilor router-ului prin care i se ofera senzorului access la internet. Al treila pas este reprezentat 
printr-o activitate care afiseaza durata si progresul procesului de instalare. La acest pas se realizeaza conexiunea completa a senzorului. La finalizeara conexiunii apare 
in activitate un buton care va redeschide activitatea ActivityWelcome, iar noul senzor adaugat va fi afisat in lista acesteia. 

\section{Modulul Broker MQTT}\label{sec:pi_mqttbroker}
Acest modul reprezinta punctul central al transmisiei in timp real al datelor achizitionate de senzor catre aplicatia Android.

Figura \ref{fig:MQTTBrokerInside} prezinta arhitectura interna a modulului MQTT Broker compusa din serverul MQTT Mosquitto oferit de organizatia Eclipse si o aplicatie 
Python.
\begin{figure}[H]
    \centering
    \includegraphics[scale=0.8]{figs/MQTTBrokerInside.png}
    \caption{Arhitectura interna a modulului MQTTBroker}
    \label{fig:MQTTBrokerInside}
\end{figure}

\subsection{Modulul Mosquitto MQTT Broker}\label{subsec:pi_mqttbroker_server}
Modulul Mosquitto MQTT Broker este instalat pe o masina virtuala cu un sistem de operare linux utilizand comanda "sudo apt-get install mosquitto mosquitto-clients -y". 
In mod implicit acesta este configurat sa comunice pe portul 1883, dar poate fi modificat accesand fisierul de configurare "/etc/mosquitto/mosquitto.conf". Pentru 
pornirea serverului trebuie executata comanda "sudo systemctl start mosquitto", iar pentru pornirea automata la deschiderea sistemului de operare linux se executa 
comanda "sudo systemctl enable mosquitto".

\subsection{Modulul Wildcard MQTT Subscriber}\label{subsec:pi_mqttbroker_wildcard}
Modulul Wildcard MQTT Subscriber este o aplicatie scrisa in limbajul de programare Python care are rolul de a intercepta toate mesajele care ajung in Broker de la dispozitivele 
cu rol de publicator si de a le transmite catre modulul RESTful Server pentru memorarea acestora in baza de date. In structura unei retele MQTT, aceasta aplicatie are un rol 
special de abonator pentru toate mesajele de date care ajung in Broker de la publicatori. Pentru a realiza aceasta abonare, aplicatia trimite o cerere de abonare catre 
Broker pentru topicul "readings/\#" care specifica abonarea pentru toate topicurile care incep cu "readings/" si contin orice altceva in rest '\#'. Mesajele de configurare 
sau alarmele transmise intre aplicatia Android si senzor nu vor fi interceptate de acest abonator, doarece topicurile respective incep cu sirul de caractere "config/" sau 
"alarm/".

Pentru implementarea aplicatiei Python este utilizata biblioteca "paho.mqtt.cient". Aceasta ofera rutine ajutatoare pentru crearea unui client cu rol de abonator sau publicator 
intr-o retea MQTT. De asemenea, ofera rutine de callback care permit realizarea anumitor actiuni atunci cand au loc evenimente precum receptia unui mesaj, connectarea cu 
success, subscriptia cu success etc. La receptia unui mesaj de la Broker se creeaza o cerere HTTP cu metoda POST care este transmisa catre modulul RESTful Server pentru 
salvarea datelor in baza de date. De la senzor este primit un mesaj care contine toti indicii de calitate a aerului intr-un format JSON, figura 
\ref{fig:Mqtt2AndroidDataFormat} prezinta acest format. Este de datoria aplicatiei sa extraga adresa MAC a senzorului din topic si sa adauge o unitate de timp in 
fus orar UTC la datele receptionate inainte de transmisia lor catre baza de date.

Figura \ref{fig:Mqtt2AndroidDataFormat} prezinta formatul JSON al mesajului transmis de catre senzor si retransmis de catre Broker inspre aplicatia Android si baza de date.  
\begin{figure}[H]
    \centering
    \includegraphics[scale=0.8]{figs/Mqtt2AndroidDataFormat.png}
    \caption{Formatul datelor transmise de modulul MQTT Broker catre Android}
    \label{fig:Mqtt2AndroidDataFormat}
\end{figure}

\section{Modulul Server RESTful}\label{sec:pi_restserver}
Modulul RESTful Server reprezinta punctul de legatura intre aplicatia Android si baza de date si intre modulul MQTT Broker si baza de date. Acesta este implementat in limbajul 
de programare Python si utilizeaza biblioteca Flask. Pentru comunicarea cu acesta este utilizat protocolul HTTP. 

\subsection{Arhitectura serverului}\label{subsec:pi_restserver_arhitectura}
Figura \ref{fig:PI_RealFlaskProjectStructure} prezinta structura fisierelor proiectului Python utilizand Flask. 
\begin{figure}[H]
    \centering
    \includegraphics[scale=0.9]{figs/PI_RealFlaskProjectStructure.png}
    \caption{Arhitectura generala a proiectului}
    \label{fig:PI_RealFlaskProjectStructure}
\end{figure}

Structura fisierelor proiectului respecta structura de baza a unui proiect Python utilizand Flask descrisa in sectiunea \ref{sec:flask}. De asemenea, respecta si 
modularizarea codului sursa conform modelului arhitectural MVC, Model-View-Controller. In urmatoarele paragrafe se va descrie fiecare modul continut in directorul 
src/ si care sunt functiile acestora.

\subsection{Descrierea modului de functionare}\label{subsec:pi_restserver_functionare}
Fisierul sensor\_data\_routes.py reprezinta punctul de intrare in aplicatie. Acesta contine o corelare intre adresele URL pe care server-ul le cunoaste si clasele model 
sub forma de resurse. O adresa URL identifica unic o resursa sau o clasa model. Orice cerere HTTP care vine din retea, de la aplicatia Android sau de la modulul MQTT Broker, 
vor fi validate in acest modul pe baza andresei URL.

Fisierul sensor\_data\_controller.py reprezinta punctul central al aplicatiei. Acesta primeste de la modulul Routes resursa care va fi accesata din baza de date, 
identifica metoda HTTP si argumentele acesteia, valideaza aceste informatii si transmite catre modulul Model interogarea specifica resursei. La primirea datelor de la 
modulul Model, acesta va transmite catre modulul View datele receptionate pentru a fi formatate corespunzator, iar apoi va transmite raspunsul catre entitatea care a transmis 
cererea.

Fisierul sensor\_data\_model.py reprezinta modulul Model al aplicatiei. Acesta contine implementarea propriu-zisa a metodelor de interogare a bazei de date. Exista o metoda 
specifica fiecarei resurse pentru a facilita dezvoltarea ulterioara si optimizarea interogarilor catre baza de date.

Fisierul sensor\_data\_view.py reprezinta modulul View al aplicatiei. Acesta are rolul de a formata datele extrase din baza de date in formatul pe care entitatea care a 
efectuat cererea il cunoaste, si anume formatul GSON. 

Figura \ref{fig:Server2AndroidDataFormat} prezinta formatul JSON al datelor trimise catre aplicatia Android de la modulul Server in urma unei interogari a bazei de 
date pentru valorile de calitate a aerului dintr-o anumita perioada. Raspunsul prezentat contine o lista cu 2 citiri.
\begin{figure}[H]
    \centering
    \includegraphics[scale=0.7]{figs/Server2AndroidDataFormat.png}
    \caption{Exemplu de raspuns la o interogare a bazei de date}
    \label{fig:Server2AndroidDataFormat}
\end{figure}

\subsection{Formatarea marcajului temporal}\label{subsec:pi_restserver_timestamp}
O problema bine cunoscuta a bibliotecii datetime din limbajului de programare Python si care a fost intampinata la formatarea datelor in modulul View este fusul orar. 
Atunci cand un sir de caractere este convertit in formatul datetime sau invers, acesta nu contine informatii legate de fusul orar, UTC sau local. Sirul de caractere 
'2024-06-16T11:23:53.015001Z' va fi interpretat ca un obiect de tipul datetime din care informatia 'Z', care reprezinta fusul orar UTC, este pierduta. 
Trebuie specificat explicit ce fus orar este utilizat la apelul functiei de convertire. Atunci cand este specificat fusul orar, sirul de caractere 
'2024-06-16T11:23:53.015001Z' va fi interpretat ca un obiect datetime din care informatia 'Z' este inlocuita cu '+00:00'. Ambele siruri, 'Z' si '+00:00' semnifica fus orar 
UTC, dar nu toate librariile din alte limbaje de programare interpreteaza corect sirul de caractere '+00:00'. In biblioteca Retrofit, modulul care mapeaza automat un sir de 
caractere in format GSON pe o clasa model nu interpreteaza corect sirul '+00:00'. Pentru a rezolva aceasta problema, in modulul View, a fost necesara inlocuirea manuala a sirului 
de caractere '00:00' cu 'Z'. O alta rezolvare a acestei problema ar fi fost scrierea manuala a codului pentru maparea datelor pe o clasa model in loc de utilizarea uneia deja 
existente in biblioteca Retrofit. 

\section{Modulul baza de date MongoDB}\label{sec:pi_bazadedate}
Acest modul are rolul de a memora datele transmise de catre senzori si de a facilita extractia acestora la cererea utilizatorilor. Acesta comunica prin intermediul 
protocolului de comunicatie TCP/IP Sockets cu serverul RESTful, acesta din urma reprezentand interfata de comunicare dintre modulele MQTT Broker si Mobile cu baza 
de date. 

Baza de date MongoDB ofera un mod eficient si optimizat de memorare a datelor care contin o unitate de timp, numit Time Series Collection. Acest lucru face ca aceasta 
baza de date sa fie cea mai potrivita pentru proiectele din domeniul IoT in care momentul de timp la care datele sunt colectate este foarte important.

\subsection{Structura bazei de date}\label{subsec:pi_bazadedate_structura}
Figura \ref{fig:PI_MongodbDocExample} prezinta structura unui document memorat in baza de date. 
\begin{figure}[H]
    \centering
    \includegraphics[scale=0.8]{figs/PI_MongodbDocExample.png}
    \caption{Exemplu de document din baza de date}
    \label{fig:PI_MongodbDocExample}
\end{figure}

Aceasta baza de date este formata dintr-o colectie numita "readings" care contine mai multe documente. Un document reprezinta datele pe care un sezor le-a achizitionat 
la un moment de timp. Structura unui document contine urmatoarele campuri:
\begin{itemize}
	\item Campul "timestamp" - reprezinta unitatea de timp la care datele au fost achizitionate de catre senzor. Aceasta unitate de timp contine an, luna, zi, ora, 
	minut, secunda si milisecunda. Formatul unitatii de timp trebuie sa respecte standardul international ISO, care defineste ordinea parametrilor de timp, 
    fusul orar si caracterele utilizate pentru delimitarea acestora.
	\item Campul "metadata" - reprezinta informatii despre senzor si despre datele continute in document. Acest camp ar trebui sa contina elemente care se schimba 
	foarte rar sau niciodata. De asemenea, elementele pe baza carora se fac interogarile in baza de date trebuie sa fie continute in acest camp pentru a beneficia 
    de eficienta maxima. Acest camp contine adresa MAC a senzorului si tipul de senzor. 
	\item Campurile "PM2.5", "PM1.0", "VOCIndex", "TPS", "PM4.0", "Humidity", "Temperature", "PM10.0" - reprezinta indicii de calitate a aerului trimis de catre senzor.  
    In dreptul acestor indici se afla valorile corespunzatoare. De exemplu, temperatura in grade celsius.
	\item Campul "\_id" - reprezinta un sir de caractere generat automat la memorarea documentului care identifica unic acest document.
\end{itemize}

\subsection{Interogarea bazei de date}\label{subsec:pi_bazadedate_interogare}
Figura \ref{fig:PI_AggregationExample} prezinta operatia de agregare utilizata pentru extragerea documentelor care se afla intr-un anumit interval de timp. Prima 
operatie efectuata este "\$match" care va selecta din baza de date toate documentele care se afla intr-un anumit interval de timp. A doua operatie efectuata este 
tot operatia "\$match" care va selecta din documentele care au indeplinit conditia primei operatii, doar documentele care au o anumita adresa MAC si care contin 
un anumit tip de valori, cum ar fi valori de umiditate. A treia operatie este "\$sort" care va aranja documentele care au indeplinit conditiile celei de-a doua 
operatii in ordine cronologica. Ordinea acestor operatii este foarte importanta pentru cresterea eficientei. S-a ales operatia de selectare a intervalului de timp 
ca prima operatie, deoarece este operatia cea mai putin costisitoare de efectuat pe intreaga baza de date. Astfel, setul de documente pe care se efectueaza celelalte 
doua operatii, care sunt mult mai costisitoare, este diminuat semnificativ.
\begin{figure}[H]
    \centering
    \includegraphics[scale=0.7]{figs/PI_AggregationExample.png}
    \caption{Exemplu de operatie de agregare}
    \label{fig:PI_AggregationExample}
\end{figure}

\section{Moulul Senzor}\label{sec:pi_senzor}
\subsection{Arhitectura modulului Senzor}\label{subsec:pi_senzor_arhitectura}
Acest modul este compus din doua componente principale, un dispozitiv care are capabilitatile de a esantiona indici de calitate a aerului si de a comunica 
printr-un mediu fara fir si un router care realizeaza legatura dintre reteaua locala si internet. Rolul acestui modul este de a realiza o conexiune cu un router,
de a achizitiona date de calitate a aerului si de a le transmite catre aplicatia Android. 

Figura \ref{fig:PI_SensorUnitDiagram} prezinta arhitectura detaliata a modulului de achizitionare a indicilor de calitate a aerului.
\begin{figure}[H]
    \centering
    \includegraphics[scale=0.5]{figs/PI_SensorUnitDiagram.png}
    \caption{Arhitectura modulului Senzor}
    \label{fig:PI_SensorUnitDiagram}
\end{figure} 

\subsection{Platforma de dezvoltare Arty Z7}\label{subsec:pi_senzor_artyz7}
De pe platforma Arty Z7 sunt utilizate cele doua porturi Pmod si 5 pini de pe conectorii J2 si J7. Senzorul HDC1080 este conectat la conectorul PmodB, iar senzorul 
SGP40 la placa de circuite integrate PmodHYGRO in conectorul pus la dispozitie pentru legarea mai multor senzori in serie. Acesti 2 senzori sunt conectati la aceeasi 
magistrala I2C si sunt diferentiati prin adresare. Controlerul de retea ATWINC1500 este conectat la portul PmodA al placii Arty Z7, iar senzorul SPS30 la conectorii J2 
si J4. Sistemul integrat Zynq-7000 reprezinta unitatea de control a tututor perifericelor de pe platforma si a portului MicroUSB utilizat pentru alimentarea cu 5 volti 
si pentru comunicatia cu un calculator personal prin interfata de comunicare UART. 

Conectorii Pmod de pe placa de dezvoltare Arty Z7 ofera pini pentru alimentarea cu 3.3V. Senzorul SPS30 trebuie alimentat la tensiunea de 5V care este disponibila pe 
conectorul J7. 

Pentru realizarea schemei de design a fost utilizat mediul de dezvoltare Xilinx Vivado 2018.3 \cite{zynq7000ug892}, iar pentru dezvoltarea codului sursa in limbajul 
de programare C a fost utilizat mediul de dezvoltare Xilinx SDK 2018.3 \cite{zynq7000ug1145}. Procesorul ARM Cortex-A9 \cite{zynq7000ug585} este utilizat pentru 
executarea instructiunilor microcontroller-ului, iar partea logica pentru conectarea porturilor necesare la microcontroller. Au fost alese mediile de dezvoltare cu 
versiunea 2018.3 din cauza unor probleme de integrare intampinate utilizand mediul de dezvoltare Vitis. Mai exact, generarea automata a librariei pe baza fisierului 
".xsa" genera multiple erori. Versiunea 2018.3 a fost aleasa pe baza experientei din trecut.

\subsection{Diagrama de proiectare a procesorului Zynq 7000}\label{subsec:pi_senzor_designzynq}
Schema de design a platformei Arty Z7 contine un set de periferice si porturi care sunt conectate la procesorul ARM Cortex-A9 si un set de periferice si porturi 
conectate la partea logica FPGA. De exemplu, porturile pmod sunt conectate la partea logica, iar portul MicroUSB este conectat la procesor. Pentru utilizarea 
componentelor sau porturilor periferice conectate la partea logica FPGA in procesorul ARM Cortex-A9 este utilizata interfata EMIO care este o extensie a 
multiplexorului de intrari si iesiri al procesorului.

Figura \ref{fig:PI_ZynqBlockDesign} prezinta interconexiunea dintre procesor si porturile periferice.
\begin{figure}[H]
    \centering
    \includegraphics[scale=0.6]{figs/PI_ZynqBlockDesign.png}
    \caption{Diagrama de proiectarea procesorului Zynq 7000}
    \label{fig:PI_ZynqBlockDesign}
\end{figure}

Portul "ATWINC\_IRQN" reprezinta semnalul de intrerupere al controllerului de retea ATWINC1500 activ in 0 logic. Acesta este conectat la o poarta NU pentru a 
inversa polaritatea semnalului, deoarece unitatea generala de control al intreruperilor (GIC) nu are polaritate configurabila. Acesta genereaza o intrerupere 
doar la tranzitia din 0 logic in 1 logic. Porturile a caror denumire incep cu "SPI0" reprezinta semnalele protocolului SPI utilizate in comunicarea cu controllerul 
de retea ATWIN1500, iar portul "GPIO" reprezinta semnalul de reset al modulului ATWINC1500 care, la fel ca semnalul de intrerupere, este activ in 0 logic. Aceste 
semnale sunt legate la portul PMODA (JA) de pe platforma Arty Z7.

Portul "IIC\_0\_0" reprezinta semnalele primei magistrale I2C utilizate in comunicatia cu senzorii HDC1080 si SPG40. Acesti senzori sunt conectati la aceeasi magistrala I2C. 
Aceste semnale sunt legate pe portul PMODB (JB) de pe platforma Arty Z7.

Portul "IIC\_1\_0" reprezinta semnalele celei de-a doua magistrale I2C utilizate in comunicatia cu senzorul SPS30. Aceste semnale sunt legate la conectorii J2 si J7 
de pe platforma ArtyZ7.

Porturile "DDR" si "FIXED\_IO" sunt generate automat la crearea unitatii de procesare. Portul "DDR" realizeaza conexiunea dintre procesorul ARM Cortex-A9 si memoria 
externa DDR utilizata la incarcarea si executarea codului sursa, iar portul "FIXED\_IO" reprezinta semnalele de intrare/iesire ale perifericelor utilizate. De exemplu 
semnalele conectate intre interfata UART si portul microUSB.

Pentru creearea diagramei de proiectare in mediul de lucru Vivado 2018.3 sunt necesari urmatorii pasi descrisi succint:
\begin{enumerate}
	\item Se creeaza un proiect nou selectand sistemul integrat Zynq7000.
	\item Din meniul "IP Integrator" se creeaza un nou bloc de design.
	\item Prin apasarea butonului "Add IP" din fereastra design-ului se adauga procesorul Zynq7000.
	\item Prin dublu click pe blocul nou aparut al procesorului se deschide meniul de configurare al acestuia. Din acest meniu se selecteaza sursa de ceas si frecventa 
	acestuia, liniile de intrare iesire si perifericele utilizate.
    \item La apasarea butonului "OK" vor aparea pe blocul procesorului toate perifericele selectate la pasul anterior care urmeaza a fi conectate la o componenta externa.
    \item Crearea porturilor externe se face prin click dreapta pe diagrama si "Create port". Acestuia i se atribuie un nume, o directie si un tip.
    \item Pe baza diagramei create Vivado va genera automat instructiunile necesare maparii porturilor prin click dreapta pe fisierul designului si "Create HDL Wrapper".
    \item Se genereaza sirul de biti pentru partea logica FPGA prin apasarea butonului "Generate Bitstream".
    \item Se exporta diagrama si sirul de caractere intr-un fisier pe care mediul de dezvoltare Xilinx SDK il va interpreta si va genera driverele necesare. Acest 
    fisier contine informatiile necesare generarii driverului doar pentru perifericele activate si configurate la pasul 4.
\end{enumerate}

Alegerea frecventei sursei de ceas este in stransa legatura cu modulele timer necesare functionarii aplicatiei, si anume, doua module timer care numara la milisecunda. 
Pentru a obtine o acuratete cat mai buna a milisecundei, sursa de ceas este multiplicata cu un factor de 16, obtinand astfel o frecventa a sursei de ceas de 400Mhz.

\subsection{Structura proiectului de pe Zynq 7000}\label{subsec:pi_senzor_structurasdk}
Driverul generat automat de catre mediul de dezvoltare Xilinx SDK se afla intr-un proiect separat numit "zynq7000\_atwinc\_sdk\_bsp". Acest proiect este generat automat 
bazat pe fisierul ".hdf" exportat de mediul de dezvoltare Vivado.

Proiectele necesare pentru incarcarea codului sursa in memoria flash a procesorului sunt separate de proiectul care contine implementarea efectiva a functionalitatii 
senzorului. Acestea se numesc zynq7000\_atwinc\_sdk\_FSBL\_bsp si zynq7000\_atwinc\_sdk\_FSBL si sunt generate automat de mediul de dezvoltare Xilinx SDK. 

Figura \ref{fig:PI_SDKFileStructure} prezinta structura fisierelor proiectului in mediul de programare Xilinx SDK. Fiecare director reprezinta un modul al proiectului 
si contine fisere asociate cu denumirea acestuia. Structura fisierelor a fost impartita in doua imagini asezate in paralel pentru a reduce inaltimea acesteia. Fisierele 
prezentate in imaginea din dreapta se afla sub directorul "src" exceptand fisierele "main.c" si "lscript.ld".
\begin{figure}[H]
    \centering
    \includegraphics[scale=0.60]{figs/PI_SDKFileStructure.png}
    \caption{Structura fisierelor proiectului}
    \label{fig:PI_SDKFileStructure}
\end{figure}

Directorul "Application" contine fisiere sursa si antet specifice aplicatiei. Fisierul "acquisition.c" contine rutine pentru initializarea si gestionarea achizitiei 
de date de la senzorul de temperatura si umiditate HDC1080. Acesta contine o masina de stari care initiaza o noua achizitie de date, asteapta finalizarea acesteia, 
converteste valorile obtinute in sir de caractere si gestioneaza posibilele erori. Fisierul "acquisition.h" contine declaratia starilor masinii de stari si a 
variabilelor si functiilor utilizate de fisiere externe. Fisierul "publish.c" este responsabil cu calcularea urmatorului moment in care se va transmite un pachet,
de creearea pachetului si adaugarea acestuia in coada. Fisierul "app.c" contine masina de stare generala a aplicatiei, functii de tip handler care gestioneaza 
evenimentele care au loc in retea, masina de stare pentru resetarea dispozitivului la setarile din fabrica si defineste o coada de mesaje si rutinele necesare 
pentru gestiunea acesteia, adaugare pachet, extragere pachet, stergere pachet. 

Directorul "atwinc15x0" reprezinta driverul pentru controlerul ATWINC1500 oferit de organizatia Microchip sub licenta libera. Driverul este inclus in libraria 
"mla", Microchip Libraries for Applications, si poate fi descarcat de pe site-ul oficial al organizatiei Microchip. Acesta este descris in detaliu in subcapitolul
\ref{subsec:af_atwinc}.

Directorul "http" contine rutinele necesare crearii si mentinerii unei conexiuni HTTP. Aceste rutine sunt utilizate la transmiterea informatiilor intre aplicatia 
Android si senzor printr-o conexiune directa in procesul de instalare al senzorului.

Directorul "MQTT" contine rutinele necesare crearii si mentinerii unei conexiuni cu un Broker MQTT si formatarea pachetelor conform acestui protocol. Implementarea 
driverului este oferita de organizatia Eclipse cu liceta deschisa si poate fi descarcat sub forma de arhiva de pe github. Acest driver este impartit in doua directoare, 
"MQTTClient-C" si "MQTTPacket". Directorul "MQTTClient-C" contine rutinele necesare crearii si mentinerii conexiunii, iar "MQTTPacket" contine rutine pentru serializarea 
diferitelor pachete specifice protocolului MQTT. Fisierul antet din directorul "MQTTClient-C" contine declaratii externe ale rutinelor de scriere si citire a unui pachet si 
ale rutinelor pentru managementul timerelor care sunt specifice fiecarei platforme si trebuie implementate de catre integratorul driverului. Driverul este compatibil cu 
mai multe platforme, Linux, Windows, EmbeddedOS sau Embedded. Functionalitatea oferita pentru platforma Embedded este bazata pe asteptare ocupata, firul de executie 
este blocat pentru asteptarea raspunsului pentru un anumit pachet, care este o practica contraindicata in sistemele embedded. Avand in vedere acest lucru, fisierele 
din directorul "MQTTClient-C" au fost modificate astfel incat sa nu mai blocheze firul de executie. Modificarile constau in inlocuirea zonelor de cod unde se asteapta 
receptionarea unui raspuns la pachetul transmis sau expirarea timpului acordat pentru operatia curenta cu o variabila de stare si un modul timer global care este 
verificat periodic la frecventa rularii buclei principale. Variabila de stare contine statusul curent al modulului MQTT, operatie in derulare sau liber pentru 
o noua operatie si functioneaza ca un semafor, daca exista o operatie in derulare nu se va initia o noua operatie si la urmatoarea executie a buclei principale se va 
verifica din nou daca este posibila initierea noii operatii. Timerul global se asigura ca daca o operatie in derulare nu ofera nici un raspuns intr-o perioada 
predefinita, acesta va executa rutinele necesare rezolvarii erorii care a dus la pierderea operatiei. La rutinele din directorul "MQTTPacket" nu au fost ecesare 
modificari, fiind complet decuplate de platforma de executie. 

Directorul "Interfaces" reprezinta o abstractizare a driverelor utilizate in proiect, Zynq driver, ATWINC1500 driver si MQTT driver. Fiecare fisier sursa din acest director 
contine rutine pentru initializarea perifericului specific denumirii fisierului si rutine pentru controlul acestuia. De asemenea, in cazul perifericelor care functioneaza 
bazat pe intreruperi fisierul contine si rutinele de tip Handler care se executa la aparitia unui eveniment. Fisierele antet contin macrouri pentru caracteristicile 
fiecarui periferic, macrouri pentru identificarea pinilor sau pentru functii care pot fi scrise intr-o singura linie de cod, si declaratii ale rutinelor care sunt 
utilizate de fisiere externe. Fisierele utils.c si utils.h contin rutine generale utile intr-un proiect, cum ar fi o rutina pentru blocarea executiei pentru o anumita 
perioada de timp (BusyDelay) si rutine pentru conversii intre tipuri de veriabile, de exemplu conversia din float in string sau inversarea ordinii caracterelor intr-un 
string. Fisierele "netwrok\_protocol.c" si "network\_protocol.h" abstractizeaza operatiile necesare pentru initializarea si crearea unui Socket si pentru conectarea 
la server.

Directorul "Peripherals" contine fisierele specifice fiecarui periferic al platformei Zynq Z7. Fisierele "hdc1080.c" si "hdc1080.h" definesc rutinele pentru initializarea 
si configurarea modulului HDC1080 si o masina de stare pentru inceperea unei esantionari de date si citirea acestora. Fisierele "sgp40\_i2c.c" si "sgp40\_i2c.h" specifice 
senzorului SGP40 si fisierele "sps30.c" si "sps30.h" specifice senzorului SPS30 reprezinta un nivel de abstractizare al driver-ului oferit de Sensirion pentru acestia. 
Rutinele continute in aceste fisiere corespund cu comenzile care pot fi trimise catre senzori, cum ar fi obtinerea numarului serial de indentificare, pornirea unei achizitii, 
pornirea procesului de calibrare automata etc, iar structurarea datelor conform cu structura registriilor senzorilor si comunicatia peste protocolul I2C este realizata de 
catre fisierele din directorul "sensirion\_driver".  

Directorul "sensirion\_driver" reprezinta driver-ul pentru senzorii SGP40 si SPS30. Fisierele "sensirion\_i2c.c" si "sensirion\_i2c.h" contin functii de tip stub pentru 
initializarea, scrierea si citirea peste interfata de comunicatie I2C. Aceste functii sunt de tipul stub, sunt specifice platformei in care este utilizat driver-ul si 
trebuie implementate de catre integrator. Fisierele "sensirion\_common.c" si "sensirion\_common.h" contin functii pentru diferite conversii intre tipuri de date, cum 
ar fi conversia din float in sir de octeti, functii pentru calcularea si verificarea octetului de verificare a integritatii pachetului (CRC) si functii pentru formatarea 
datelor care sunt scrise in sau citite din registrii celor 2 senzori SGP40 si SPS30 conform cu structura acestora.

Directorul "sensirion\_gas\_index\_algorithm" contine algoritmul pentru interpretarea datelor citite de la senzorul SGP40.

Fisierul lscript.ld reprezinta fisierul care defineste adresa de baza a memoriei unde este scris codul sursa, dimensiunea memoriilor RAM si dimensiunea stivei.

\subsection{Diagrama de stari a proiectului de pe Zynq 7000}\label{subsec:pi_senzor_statediagram}
Figura \ref{fig:PI_MainStateMachine} prezinta principalele sarcini si evenimente ale senzorului.
\begin{figure}[H]
    \centering
    \includegraphics[scale=0.66]{figs/PI_MainStateMachine.png}
    \caption{Masina de stare principala}
    \label{fig:PI_MainStateMachine}
\end{figure}

Urmatoarele paragrafe contin o descriere a fluxului de executie prezentat in figura \ref{fig:PI_MainStateMachine} cu accentul pe functionalitatile care nu au putut 
fi cuprinse in figura din cauza complexitatii si a dimensiunii. 

La pornirea sistemului este executata starea "Initialize all peripherals" unde sunt initializate toate perifericele utilizate de catre aplicatie, atat cele externe 
platformei Arty Z7, modulele HDC1080, SGP40, SPS30 si ATWINC1500, cat si cele integrate, modulele Clock, Timer, Interrupts etc.

La pornirea buclei infinite este executata starea "Search for AP information in memory" in care se verifica daca modulul ATWINC1500 contine in memorie informatiile 
necesare pentru conectarea la un router. Aceste informatii sunt numele retelei (SSID), tipul de autentificare si parola, fiind memorate la finalul procesului de instalare. 
Daca nu exista informatii salvate in memorie, se va executa ramura din dreapta a figurii \ref{fig:PI_MainStateMachine}, ramura "Nu" a conditiei "Have saved AP info?". 
Aceasta ramura prezinta pasii care sunt executati in procesul de instalare al senzorului.

Procesul de instalare al senzorului incepe cu starea "Enter AP Mode" unde modulul ATWINC1500 intra in starea de AP (Access Point). In aceasta stare dispozitivul 
functioneaza ca un router care accepta o singura conexiune. Apoi, in starea "Wait user connection", modulul ATWINC1500 asteapta ca aplicatia Android sa se conecteze 
la el. Dupa realizarea conexiunii se executa starea "Listen for Socket Conn" unde modulul ATWINC1500 creeaza un socket si asteapta ca aplicatia Android sa se conecteze 
la acest socket pentru a putea comunica. Dupa realizarea conexiunii la socket, in starea "Exchange router info" senzorul primeste informatiile necesare pentru conectarea 
la un router. Aceste informatii sunt salvate in memorie in starea "Save router info", iar apoi se intra pe fluxul de executie al cazului in care informatiile routerului 
exista in memorie.

In starea "Connect to router" este realizata conexiunea senzorului la router bazat pe informatiile din memorie. 

Dupa conectarea cu succes la router este executata starea "Initialize MQTT module" unde este initializat driverul MQTT. In aceasta stare se aloca zonele de memorie 
utilizate pentru transmisia si receptia de pachete si timpul de asteptare pentru receptia unui raspuns.

In starea "Connect Socket" se creeaza un canal de comunicatie si se realizeaza conexiunea cu masina pe care ruleaza Brokerul MQTT.

Dupa conectarea cu success la platforma in care ruleaza Brokerul MQTT, in starea "Connect to MQTT Broker", se realizeaza conexiunea cu acesta prin transmiterea 
pachetului MQTT CONNECT.

In starea "Subscribe for alarms" senzorul transmite un mesaj Brokerului MQTT prin care cere subscriptia la topicul pentru configurari. Acest topic este utilizat pentru 
receptia de mesaje de configurare de la aplicatia Android.

Daca in oricare din starile care presupun o conexiune cu un modul extern are loc o eroare se va aplica un algoritm de crestere exponentiala a timpului dintre 
reincercari. Acest algoritm incepe cu o reincercare imediata, apoi va reincerca conexiunea dupa 10 secunde, dupa 30 de secunde, dupa un minut si in final la fiecare 
5 minute. Acest algoritm previne reincercarea conexiunii prea frecvente care poate duce la supraincarcarea modulului extern.

Urmatoarea stare este ilustrata printr-un romb gol care semnifica inceperea buclei de functionare a aplicatiei. Cat timp senzorul functioneaza normal, nu se pierde 
conexiunea cu oricare din modulele externe sau nu este actionat fizic de catre utilizator, aceasta va executa aceasta bucla la infinit. In aceasta bucla, in starea 
"Queue Empty?" se verifica daca exista un mesaj in coada de aplicatie. Daca exista un mesaj acesta este transmis, iar daca nu exista se trece mai departe si se verifica, 
in starea "Publish time" daca a venit momentul pregatirii unui nou pachet. Acest moment este calculat in secunde si este mentinut intr-o variabila globala. Daca 
secunda curenta este mai mica decat secunda la care trebuie pregatit urmatorul pachet se reia bucla, altfel se executa conditia "Data Ready?". Aceasta conditie 
informeaza modulul de achizitie ca este necesar un nou set de date de calitate a aerului. Daca aceste date nu sunt pregatite, se va executa starea "Start Acquisition" 
care porneste o noua achizitie, apoi la fiecare executie a buclei se verifica daca achizitia s-a finalizat. La finalul achizitiei, este pregatit un pachet de tip 
PUBLISH in care sunt adaugate datele achizitionate, iar apoi sunt urcate in coada de aplicatie in starea "Enqueue MQTT Publish". Urmeaza starea "Compute Next publish" 
in care variabila care mentine secunda la care urmatorul pachet trebuie pregatit este calculata bazat pe secunda curenta a sistemului si pe perioada de publicare. La 
urmatoarea rulare a buclei vor fi gasite doua mesaje in coada de aplicatie si vor fi transmise catre Brokerul MQTT.
