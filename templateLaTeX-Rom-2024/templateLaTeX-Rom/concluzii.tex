\chapter{Concluzii}
\pagestyle{fancy}

Acest capitol va fi prezentat in 3 sectiuni, contributiile personale aduse domeniului monitorizarii calitatii aerului, analiza rezultatelor obtinute in urma 
dezvoltarii lucrarii de fata si posibile dezvoltari ulterioare.

\section{Contributii personale}\label{c_contributii}
Integrarea protocolului MQTT cu procesorul Zynq 7000 si imbunatatirea performantei driver-ului Paho MQTT oferit de Eclipse care este executat pe senzor prin 
inlocuirea portiunilor de cod unde se utiliza blocarea buclei sistemului in asteptarea unui eveniment cu o semaforizare bazata pe module Timer.

Integrarea controller-ului de retea ATWINC1500 cu procesorul Zynq 7000 ofera posibilitatea de a concentra puterea de procesare a cipului pe achizitia datelor de 
la senzorii de calitate a aerului si, intr-o dezvoltare ulterioara, de a executa algoritmi constisitori pentru procesarea acestor date.

\section{Analiza rezultatelor}\label{c_analiza_rezultate}
In urma testelor efectuate pe sistem s-a constatat ca vizualizarea si interactiunea cu graficele pe un ecran atat de mic cum este cel al unui telefon este dificila. La 
afisarea datelor istorice pe o perioada mai lunga, cum ar fi o saptamana sau mai mult, se pierd multe informatii din cauza afisarii acestora intr-un grafic restrans 
ca dimensiune. 

Utilizarea sistemului pe o perioada indelungata din punctul de vedere al unui utilizator care a achizitionat acest sistem a demonstrat utilitatea acestuia si necesitatea 
de a fi integrat in spatiile inchise. Observarea in a aplicatie a unor valori care prezinta o calitate a aerului slaba a generat cautarea si inlaturarea sursei problemei. 
De exemplu, necesitatea aerisirii incaperii dupa cateva ore de gatit, deoarece indexul compusilor oranici volatili a depasit pragul unei bune calitati a aerului.

\section{Dezvoltari ulterioare}\label{c_dezvoltari_ulterioare}
Imbunatatirea consumului de putere pentru adaptarea sistemului la functionarea pe baterii. In sectiunea \ref{sec:tv_pwrcons} a fost prezentat consumul senzorului de 
calitate a aerului de 150 mA/h. Scaderea acestui consum implica limitarea utilizarii led-urilor strict la procesul de instalare, activarea perifericelor de comunicatie 
I2C si SPI doar atunci cand sunt utilizate pentru o achizitie de date, implementarea modului de consum de putere redus (Low Power Mode) pe Zynq 7000, activarea modului 
de consum redus pe senzori si pe controllerul de retea atunci cand nu sunt utilizati, etc. Consumul de putere ar trebui sa ajunga la o valoare care sa permita utilizarea 
senzorului in mod continuu timp de minim un an.

Integrarea senzorului intr-un sistem comun de monitorizare a calitatii aerului. Asa cum a fost prezentat in capitolul \ref{ch:studiubib} exista mai multe sisteme care pun 
la dispozitia populatiei harta lumii impanzita de senzori de calitate a aerului a caror date pot fi vizualizate in timp real.

