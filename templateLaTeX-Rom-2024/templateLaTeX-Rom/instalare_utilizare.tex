\chapter{Manual de instalare și utilizare}
\pagestyle{fancy}

\section{Resurse necesare}\label{sec:iu_resurse_necesare}
Pentru instalarea si utilizarea aplicatiei sunt necesare urmatoarele resurse hardware si software:
\begin{itemize}
    \item Client git pentru clonarea proiectului.
	\item Senzorul fizic de monitorizare a calitatii aerului format din placa de dezvoltare ArtyZ7, controller-ul de retea ATWIC1500 si cei 3 senzori HD1080, 
	SPS30 si SGP40.
	\item O masina cu sistem de operare windows sau linux pe care sa ruleze baza de date MongoDB, broker-ul MQTT Mosquitto, server-ul RESTful pentru accesul 
	la baza de date si server-ul de legatura dintre Mosquitto si baza de date.
    \item Telefon mobil cu sistem de operare Android.
\end{itemize}

\section{Instalarea senzorului}\label{sec:iu_instalare_senzor}
Da la urmatorul link https://github.com/Coposescu/zynq7000\_atwinc.git poate fi descarcat codul sursa pentru senzor. In urmatoarele puncte sunt descrisi pasii 
pentru generarea microprogramului, incarcarea acestuia pe placa si resursele necesare pentru aceste actiuni:
\begin{enumerate}
    \item Instalarea mediului de programare Vivado HLx 2018.3 si a mediului de programare Xilinx SDK 2018.3 de pe pagina oficiala AMD -> Products -> 
    Software, Tools, \&Apps.
    \item Din mediul Vivado -> File -> Open se deschide proiectul zynq7000\_atwinc.
    \item Dupa deschiderea proiectului, in navigatorul din stanga se selecteaza "Generate Bitstream" pentru a genera microprogramul pentru partea logica FPGA.
    \item Din meniul File -> Export -> Export Hardware.. se bifeaza casuta "Include bitstream" si se selecteaza "OK" pentru crearea fisierului ".hdf".
    \item Se deschide mediul Xilinx SDK si de la meniul File -> Open Projects from File System se va deschide o fereastra in care trebuie introdusa calea catre 
    proiect. Pentru acest lucru se selecteaza "Directory" si se cauta in sistemul de fisiere folder-ul cu numele "zynq7000\_atwinc\_sdk". Acest folder se afla in 
    proiectul descarcat de pe git zynq7000\_atwinc/zynq7000\_atwinc.sdk.
    \item Se repeta pasul anterior pentru proiectele: 
    \begin{itemize}
        \item "zynq7000\_atwinc\_FSBL".
        \item "zynq7000\_atwinc\_FSBL\_bsp".
        \item "design\_1\_wrapper\_hw\_platform\_0".
        \item "zynq7000\_atwinc\_sdk\_bsp".
    \end{itemize}
    \item In meniul "Project Explorer" afisat in stanga ecranului ar trebui sa apara cele 5 proiecte. Prin click dreapta pe fiecare proiect se selecteaza din meniul aparut 
    "build" pentru generarea microprogramului specific fiecarui proiect.
    \item Dupa ce toate proiectele sunt construite cu success se acceseaza meniul Xilinx -> Create Boot Image pentru crearea microprogramului final care va fi incarcata pe 
    placa Arty Z7.
    \item In fereastra care sa deschis se bifeaza casuta "Import from existing BIF file", se introduce calea catre acest fisier 
    "zynq7000\_atwinc\_FSBL\\bootimage\\zynq7000\_atwinc\_FSBL.bif" si se selecteaza "Create Image".
    \item Pe placa de dezvoltare Arty Z7 jumper-ul JP4 trebuie sa fie pozitionat in modul JTAG. Dupa pozitionarea corecta a jumper-ului placa trebuie resetata prin deconectarea 
    si reconectarea alimentarii.
    \item Din meniul Xilinx -> Program Flash -> Program se incarca microprogramul final pe placa.
    \item Jumper-ul JP4 trebuie mutat in pozititia QSPI si resetata placa din alimentare din nou.
    \item Ledurile LD13 si DONE ar trebui sa fie aprinse daca toti pasii au fost respectati.
\end{enumerate}

\section{Instalare servere}\label{sec:iu_instalare_servere}
\subsection{Instalare Mosquitto}\label{subsec:iu_instalare_mosquitto}
\subsection{Instalare MongoDB}\label{subsec:iu_instalare_mongoDB}
\subsection{Instalare programe Python}\label{subsec:iu_instalare_python}
\section{Instalare aplicatie Android}\label{sec:iu_instalare_app_android}
\section{Utilizarea intregului sistem}\label{sec:iu_utilizare_sistem}
